%% ------------------------------------------------------------------------ %% 
\chapter{¿Qué es la inteligencia de decisiones?}
\label{Cap1}
Regreso un poco al tema sobre toma de decisiones e inteligencia de DECISIONES. 


¡Atención! No BI sino DI Decision Intelligence. 


¿Qué es inteligencia de decisiones? 


Es transformar información en mejores acciones a gran escala en cualquier rubro. 


¿Qué plantea? 


La creciente velocidad en la creación de data ha hecho que la necesidad de tomar buenas decisiones es cada vez más y más importante. Suena obvio, ¿no? 


Lo que no es obvio es que menciona que no debes ser el Data scientist más inteligente del lugar sino un buen tomador de decisiones. Trabajar con múltiples perfiles de datos y usar todos los pedazos de información e integrarlo en un mapa de decisiones. 


¿Para qué?


Para que puedas lograr tu potencial como un real líder que es Data Driven. 


Un tema interesante que se menciona es que tradicionalmente las empresas han abordado la toma de decisiones como algo que simplemente ocurre de forma orgánica o por inspiración. 


(Obviamente que no es así)


Entonces ¿Cómo debería ser?


Debería ser una competencia que si la practicas te puedes volver cada vez mejor si lo haces de forma correcta. 


Nota: en este caso las definiciones son de la propuesta teórica de Cassie Kozyrkov la primer Chief decision scientist de Google. 


%Comparto un link de un Q&A de DI:  %https://www.youtube.com/watch?v=pRtGqfYLCFk


Otros textos interesantes en teoría de decisiones puedes encontrar en ramas como: economía, neurociencias y obviamente psicología. 


Fast and slow de Daniel Kahneman y aquí los invito a leer no "como una curiosidad" sino como algo que podemos aplicar en  nuestro día a dia. 


Calkulaba


%#decisionintelligence #inteligenciadedecisiones #data #analytics
