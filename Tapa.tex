\documentclass[a4paper,12pt]{extbook}
\usepackage{fancyhdr}

\usepackage{geometry}
\usepackage{xcolor}
\usepackage{amsmath}
%\usepackage[some]{background} %From package background
%\usepackage[nottoc,notlot,notlof]{tocbibind}

\usepackage{epigraph}
\usepackage{geometry}
\usepackage{amsmath}

\usepackage{caption}
\usepackage{subcaption}
\usepackage{multirow}

\usepackage{graphicx}
\usepackage{times}
\usepackage{titlesec}
\newcommand{\sectionbreak}{\clearpage}
\usepackage{setspace}
\onehalfspacing
\usepackage[utf8]{inputenc}%tildes
\date{}%sin fecha
\usepackage{paralist}%listas
\usepackage[spanish, mexico]{babel}%spanish
%\usepackage{csquotes}
%\usepackage[style=ieee]{biblatex}

\usepackage{emptypage}
\usepackage{multirow}
\usepackage{enumerate} %enumerate
\usepackage{icomma}%coma para decimales
\usepackage{authblk}%autores 
\usepackage{float}
\usepackage{caption}


\usepackage[pdfborder={0 0 0}, bookmarksopen=true, breaklinks, hyperfootnotes=false]{hyperref}
\usepackage {bookmark}
\makeatletter
%\hypersetup{pdftitle=\@title, pdfauthor=\@author}
\makeatother
%\usepackage[all]{hypcap}
\makeatletter
\renewcommand\@seccntformat[1]{}
\makeatother

\pagestyle{empty}

\usepackage{tabularx}


%\fancypagestyle{plain}{%
\fancyhf{} % clears headers and footers
\renewcommand\headrulewidth{0pt} % no line between document and header
\fancyfoot[EL]{\thepage}% Footer on even pages (E), left (L)
\fancyfoot[OR]{\thepage}% Footer on odd pages (O), right (R)
%\renewcommand{\headrulewidth}{0pt}%
%\renewcommand{\footrulewidth}{0.4pt}
%}% Sets footer rule to 0pt

\pagestyle{fancy}

%referencias
%\usepackage{apacite}
\usepackage{cite}


%\renewcommand{\BBAA}{&}  % between authors in parenthetical cites and ref. list
%\renewcommand{\BBAB}{&}  % between authors in in-text citation
%\renewcommand{\BAnd}{&}  % for ``Ed. \& Trans.'' in ref. list

%Image path
%\graphicspath{{"/Users/jsaraviadrago/Documents/MP blog 2021/Book_data/Data_graphs/Graphs/"}}


%Margenes
\usepackage{geometry}
\geometry{a4paper, tmargin=1.5cm, lmargin=1.5cm, rmargin=1.5cm, bmargin=1.5cm}
%\geometry{a4paper, tmargin=2.5cm, lmargin=2.5cm, rmargin=2.5cm, bmargin=2.5cm, includehead} 
%Usar si titulos tienen poco margen arriba


\setlength\parindent{0.5in} %sangria parrafos
\titleformat{\chapter}{\normalfont\Large}{\Large \bf }{20pt}{\Large \bf} 
\titleformat*{\section}{\large\bfseries} % \center\bfseries con centrado de título
\titleformat*{\subsection}{\large\bfseries}
\titleformat*{\subsubsection}{\large\itshape}{}{}{}
\titleformat*{\paragraph}{\large\bfseries}
\titleformat*{\subparagraph}{\large\bfseries}
\titlespacing*{\section}{0pt}{0pt}{12pt}
\titlespacing*{\subsection}{0pt}{*5}{12pt}
\titlespacing*{\subsubsection}{0pt}{*5}{12pt}


%\definecolor{titlepagecolor}{cmyk}{1,.60,0,.40}

%\backgroundsetup{
%scale=1,
%angle=0,
%opacity=1,
%contents={\begin{tikzpicture}[remember picture,overlay]
% \path [fill=titlepagecolor] (current page.west)rectangle (current page.north east); 
% \draw [color=black, very thick] (5,0)--(5,0.5\paperheight);
%\end{tikzpicture}}
%}

\makeatletter                   
\def\printauthor{%                  
    {\large\@author}}          
\makeatother

\author{%
    \textbf{Juan Carlos Saravia}\\
    %Twitter: JCsaraviad  \\
    %\texttt{jsaraviadrago@gmail.com}\vspace{30pt} \\
    }

\linespread{1.213}

\begin{document}

\pagenumbering{gobble}
\hypersetup{pageanchor=false}

\begin{titlepage}
%\includegraphics[scale = 0.40]{C:/Users/JuanCarlosSaraviaDra/Dropbox/Tempo/images}\par\vspace{0cm}
\newgeometry{left=3cm,right=3cm,bottom=2cm}
\vspace*{0.1\textheight}
\noindent
\center
\textcolor{black}{\LARGE\textbf{\textsf{Un cuento de fútbol}}}
\vspace*{0.5cm}\par
\noindent
{\LARGE\itshape \printauthor\par}
\vfill

\begin{figure}[ht]

\centering
%\includegraphics[width=10cm, height=5cm]{C:/Users/JuanCarlosSaraviaDra/Dropbox/Tempo/images}
\caption{}
\end{figure}


\end{titlepage}
\restoregeometry




%\input{prefacio}
%\input{Agradecimientos}

%\tableofcontents
\clearpage
\hypersetup{pageanchor=true}
\pagenumbering{arabic}


%\input{Introduccion}




%\clearpage
%\thispagestyle{plain}
%\par\vspace*{.35\textheight}{\LARGE\centering \textbf{
%La versión del mismo cuento con y sin IA} \par}

%% ------------------------------------------------------------------------ %% 
\chapter{¿Qué es la inteligencia de decisiones?}
\label{Cap1}
Regreso un poco al tema sobre toma de decisiones e inteligencia de DECISIONES. 


¡Atención! No BI sino DI Decision Intelligence. 

¿Qué es inteligencia de decisiones? 

Es transformar información en mejores acciones a gran escala en cualquier rubro. 

¿Qué plantea? 

La creciente velocidad en la creación de data ha hecho que la necesidad de tomar buenas decisiones es cada vez más y más importante. Suena obvio, ¿no? 

Lo que no es obvio es que menciona que no debes ser el Data scientist más inteligente del lugar sino un buen tomador de decisiones. Trabajar con múltiples perfiles de datos y usar todos los pedazos de información e integrarlo en un mapa de decisiones. 

¿Para qué?

Para que puedas lograr tu potencial como un real líder que es Data Driven. 

Un tema interesante que se menciona es que tradicionalmente las empresas han abordado la toma de decisiones como algo que simplemente ocurre de forma orgánica o por inspiración. 

(Obviamente que no es así)

Entonces ¿Cómo debería ser?

Debería ser una competencia que si la practicas te puedes volver cada vez mejor si lo haces de forma correcta. 

Nota: en este caso las definiciones son de la propuesta teórica de Cassie Kozyrkov la primer Chief decision scientist de Google. 

%Comparto un link de un Q&A de DI:  %https://www.youtube.com/watch?v=pRtGqfYLCFk

Otros textos interesantes en teoría de decisiones puedes encontrar en ramas como: economía, neurociencias y obviamente psicología. 

Fast and slow de Daniel Kahneman y aquí los invito a leer no "como una curiosidad" sino como algo que podemos aplicar en  nuestro día a dia. 

%%%%%%%%%%%%%%%%%%%%%%%%%%

¡Un post más sobre inteligencia de decisiones!


Nota: adicionalmente te recomiendo otro libro excelente relacionado a tomar decisiones (DI).


Según Cassie el concepto más importante de análisis de decisiones es la diferencia entre una decisión y un resultado


¿Cuál es la diferencia?


Decisión:


Es una asignación irrevocable de recursos. Recurso entendido desde la forma más amplia, lo importante acá es que cuando escoges algo la otra opción desaparece. 


Resultado: 


Literalmente es la consecuencia de haber tomado esa decisión.


¿Por qué es importante? 


Porque podemos tomar una buena decisión que luego el resultado no sale tan bien o viceversa, una mala decisión que al final el resultado sale airoso. Pero este proceso está influenciado por algo que le llaman sesgos. 


Los sesgos:


En psicología los sesgos cognitivos son procesos mentales que nos permitan ahorrar energía mental. El problema que tienen es que hace que podamos generalizar o ignorar pedazos de información. 


Por ejemplo, el sesgo de resultado es evaluar una decisión basado en el resultado que se obtuvo y no en la decisión en si. ¿Cuál es el peligro? Que aprendas a tomar decisiones basado en el resultado cuando puede haber sido una mala decisión. 


El resultado tiene 2 componentes:


1. Calidad de la decisión 

2. Suerte o simplemente azar. 


Si solo te enfocas en el resultado mas no en el contexto y lo que se sabías en el momento que se realizó la decisión, puedes terminar castigando o premiando a alguien por simple suerte, por la decisión que tomaste o por los dos. Lo importante es que no sabes cuánto es de cada uno. 


Te dejo aqui otro link de una conferencia de Cassie: https://www.youtube.com/watch?v=iLu9XyZ55oI


Por otro lado, un texto que da justo al punto sobre este tema es Risk Savy, este libro es especialmente interesante porque además te pone ejemplos de cómo la gente puede leer mal los datos y a su vez esto llevarlo a tomar malas decisiones. 








%#decisionintelligence #inteligenciadedecisiones #data #analytics

%\input{historia2}
%\input{Cap1_1}
%\input{Cap1_2}
%\input{Cap1_3}
%\input{Cap1_4}

%\clearpage
%\thispagestyle{plain}
%\par\vspace*{.35\textheight}{\LARGE\centering \textbf{
%Goles, equipos e historias} \par}

%\input Cap2_2
%\input{Cap2_3}
%\input{Cap2_4}
%\input{Cap2_5}

%\clearpage
%\thispagestyle{plain}
%\par\vspace*{.35\textheight}{\LARGE\centering \textbf{
%Las aventuras dentro de la cancha} \par}

%\input{Cap3_1}
%\input{Cap3_2}
%\input{Cap3_3}


%\clearpage
%\thispagestyle{plain}
%\par\vspace*{.35\textheight}{\LARGE\centering \textbf{
%Tu guía para el Mundial de Catar 2022} \par}

%\input{cap4}
%\input{cap5}


%\appendix
%\include{ape1}

%\clearpage
%\thispagestyle{plain}
%\par\vspace*{.35\textheight}{\LARGE\centering \textbf{
%Perú historia de un grupo de guerreros } \par}

%\input{cap5_1}
%\input{epilogo}
%\input{epilogo_2} 
%\input{epilogo_3}
%\input{epilogo_final}      

% Bibliografía
%\backmatter \singlespacing   % espacio simple


%\bibliography{bibliografia}
%\bibliographystyle{plain} % arreglar esto

%\clearpage
%\thispagestyle{plain}
%\par\vspace*{.35\textheight}{

%El libro \textit{El Fútbol es Matemáticamente Posible} tiene como objetivo juntar la pasión por el fútbol con el mundo del análisis de datos. 
%El espíritu del libro es enamorar y sorprender a los fanáticos con estadísticas y datos escondidos que jamás se podrían haber imaginado.  
%Invita al lector a tener una experiencia más completa del fútbol, descubriendo anécdotas de los mundiales dignas de compartir con los demás.  Los resultados de los mundiales cobrarán vida y se entregarán a la historia del fútbol.
%Este libro es mucho más que una simple recopilación de datos. Busca ser una guía imperdible de los mundiales. Una colección de historias, tácticas y pasión futbolística sustentada con estadística que permitirán cuestionar la forma en que vemos el fútbol hoy.
%Las selecciones mundialistas utilizan estrategias diversas que a simple vista no notamos, los datos nos pueden dar luces de ellas. Este libro busca convertirse en una lámpara para ver esas estrategias, rompiendo creencias sobre equipos y sorprendiendo al lector con hallazgos numéricos impensados. De este modo, demostrar que el \textit{Fútbol es Matemáticamente Posible}.
%\par}

\end{document}